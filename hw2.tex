% aa405 introduction to aerospace plasmas hw2 due 10/10/2014 
\documentclass{article}

\usepackage[english]{babel}
\usepackage[utf8]{inputenc}
\usepackage{amsmath}
\usepackage{amssymb}
\usepackage{textcomp}
% change margins to 3/4"
\usepackage[margin=0.75in]{geometry}

\author{Jesse Dumas}
\title{AA405 Homework 2: Electromagnetism Strikes Back}
\date{\today}

\begin{document}

\maketitle

\paragraph{Problem 1}

% Biot-Savart equation
\begin{equation}
\vec{B}(r) = 
\frac{\mu_{0}}{4\pi} 
\int \frac{\vec{J}(r^{\prime})
\times \hat R}{|R|^{2}} \,
\mathrm{d}V
\end{equation}
% take curl of each side
\begin{equation}
\nabla \times \vec{B}(r) = 
\frac{\mu_{0}}{4\pi} \int \nabla \times 
\frac{\vec{J}(r^{\prime})\times 
\hat R}{|R|^{2}} \, 
\mathrm{d}V
\end{equation}
% make a substitution
\begin{equation}
\frac{\hat R}{\lvert R \rvert^{2}} = -\nabla^{2} \frac{1}{R}
\end{equation}

\paragraph{Problem 2}
\subparagraph{Question} What is the ratio of the electrostatic force of repulsion between two electrons to their gravitational force of attraction? Should this effect have been accounted for in Coulomb's experiment?

\subparagraph{Given} 
Coulomb's Law: 
\begin{equation}
\vec{F}_{2\,1\,C} = 
\frac{q_{1}q_{2}\hat{r}_{2\,1}}
{4 \pi \epsilon_{0} \lvert r_{2\,1} \rvert^{2}}
\end{equation}
Newton's law of universal gravitation: 
\begin{equation}
\vec{F}_{2\,1\,G} = 
G\frac{m_{1}m_{2}\hat{r}_{2\,1}}
{\lvert r_{2\,1} \rvert^{2}}
\end{equation}

Constants: 
$e = 1.6022 \times 10^{-19}\,C$,
$m_{e} = 9.1094 \times 10^{-31}\,kg$,
$G = 6.6726 \times 10^{-11}\,m^{3}\,s^{-2}\,kg^{-1}$,
$\epsilon_{0} = 8.8542 \times 10^{-12}\,F\,m^{-1}$

\subparagraph{Answer}
Divide Coulomb's law by Newton's law to find the ratio 
$\frac{\vec{F}_{2\,1\,C}}{\vec{F}_{2\,1\,G}}$

\begin{equation}
\frac{\vec{F}_{2\,1\,C}}{\vec{F}_{2\,1\,G}} =
\frac{
   \frac{q_{1}q_{2}\hat{r}_{2\,1}}
    {4 \pi \epsilon_{0} \lvert r_{2\,1} \rvert^{2}} 
}
{G\frac{m_{1} m_{2} \hat{r}_{2\,1}}
{\lvert r_{2\,1} \rvert^{2}}}
\end{equation}

Canceling like-terms reduces the equation to

\begin{equation}
\frac{\vec{F}_{2\,1\,C}}{\vec{F}_{2\,1\,G}} =
\frac{q_{1}q_{2}}{G 4 \pi \epsilon_{0} m_{1} m_{2}}
\end{equation}

Plugging in our constants leads to the final answer

\begin{equation}
\frac{\vec{F}_{2\,1\,C}}{\vec{F}_{2\,1\,G}} =
\frac{e^{2}}{G 4 \pi \epsilon_{0} m_{e}^{2}} =
4.1668 \times 10^{42}
\end{equation}

Therefore, there is no need to account for gravity's minuscule effect.

\paragraph{Problem 3}



\paragraph{Problem 4}



\paragraph{Problem 5}

% references, American Institute of Physics style = phaip.bst
% \bibliographystyle{phaip}
% \bibliography{ref}

\end{document}